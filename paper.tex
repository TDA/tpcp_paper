\documentclass[11pt]{article}

\usepackage{graphicx}
\usepackage{epsfig}
\usepackage{url}
\usepackage[english]{babel}
\usepackage{vmargin}
\usepackage{times}
\usepackage{amssymb}
\usepackage[fleqn]{amsmath}
\usepackage{cite}
\usepackage{titling}
\usepackage{color}


\widowpenalty=10000
\clubpenalty=10000


\begin{document}

\title{Total Platform Cyber Protection}
\date{}

\maketitle

\section{Introduction}

Traditional security defenses for the computing environment have
focused on securing the \emph{boarder} of the network, as this is the
entry place of most attacks. Specifically, technologies such as
firewalls, intrusion detection systems, and automated threat forensics
operate at the network gateway, attempting to effectively secure the
boarder of the network.

However, even with such technologies in place, there is a rash of
security breach and compromises. Target. OPM. Trump Industries.

The goal of this work is to focus research effort on \emph{securing
  the entire computing platform.} An attack must effectively target a
specific vulnerability in a specific layer of the computing stack, and
an attacker uses that vulnerability to establish persistence on the
machine, potentially attacking the underlying computing layers of the
same machine or attacking other machines. Therefore, if we wish to
increase the security of our computing systems, and reduce the number
and scope of security breaches, it is essential that we encourage
focus on novel ideas, algorithms, and techniques to secure every level
of the computing stack.

In this paper, we will first discuss the computing stack, and, for
every layer, we will identify focus areas that attempt to secure that
layer. For each of those focus areas we will answer the following
questions:

\begin{itemize}
  \item What's the technical challenge for the focus area?
  \item Why is it hard?
  \item What are the open problems?
  \item How is the area addressed in the past (from a cyber perspective)?
  \item Are there intellectual tools that need to be brought to bear on the problem?
  \item How/when do we achieve results?
  \item Near-term and the long-term?
\end{itemize}

\section{Background}

What is the background of the computing layers? Here, having the
layers in a graphic, explaining the interconnected layers would be
good.

\section{Hardware}

Important problems include how to deal with counterfeit chips, better
non-destructive examination methods, continuous monitoring of the
integrity not just at boot time, and detection of backdoors and 
``undocumented features''.  Hardware design can also be done to
aid this process, e.g., having some part of the chip consume more power
to make it more identifiable for authentication purposes.  Other
possibilities include running multiple chips with different
implementations side-by-side and comparing results to detect unexpected
behavior.

Identification and signatures using features that are more robust to
aging/use over time

\subsection{Focus Areas}

\begin{enumerate}
	\item Non-destructive IC examination, identification, and
	authentication
	\item Continuous or near-continuous physical monitoring for
	establishing root of trust
	\item Inspection-friendly design for hardware transparency
\end{enumerate}

\section{Firmware}

Firmware must deal with the fact that no OS-style abstraction layer
is available, so hardware interaction is important.  Reverse
engineering is an interesting at the firmware level because most
vendors want to make it difficult to protect IP; it is also more
tedious and would the RE process would benefit more from better tools.

\subsection{Focus Areas}

\begin{enumerate}
	\item Decouple firmware from the hardware (emulation) to aid analysis
	\item Better tools for automated reverse engineering, including
	analysis of layouts, modules, determining code vs. data
	\item How to patch firmware with analysis and guarantees about
	system behavior
	\begin{enumerate}
		\item did you fix the vulnerabilities?
		\item did you introduce new vulnerabilities?
		\item did your changes disrupt the timing of the original firmware?
	\end{enumerate}
\end{enumerate}

\section{Bus and Interconnect}

\subsection{Focus Areas}

Critical tech challenges:
\begin{enumerate}
	\item Lightweight device authentication that is bus-appropriate
	\item Operational mode setting to determine allowed capabilities
	for each device
\end{enumerate}


\section{Hypervisor}

\subsection{Focus Areas}

Critical tech challenges:
\begin{enumerate}
	\item Protect private information within a VM/guest from spying
	and interception by the host, bootstrap image with private key
	\item Attest memory, e.g., did my cloud provider put something in
	to read memory?
	\item Achieve predictable real-time scheduling in a lightweight
	and secure hypervisor
\end{enumerate}

\section{Operating Systems}

Discussion centered on reducing the attack surface by stripping out
unnecessary and unneeded functionalities.

\subsection{Focus Areas}

\begin{enumerate}
	\item Automated reduction and pruning of features to a minimal set
	of required functionality
	\item Non-control data flow attacks
	\item Concurrency-based attacks
\end{enumerate}

\section{Middleware / Runtime}

This focus of this layer includes the runtime environments and abstractions
below a high-level programming language that convert it into bytecode.

\subsection{Focus Areas}

\begin{enumerate}
	\item Dealing with bloat and reducing the attack surface
	\item Instrumentation is easier at this layer, but faster ways to observe execution are needed
	\item Controlling easily misued middleware APIs (confused deputy)
\end{enumerate}

\section{Application Layer}

Much current work focuses on just looking at memory corruption bugs, but
we need to look at how the same sort of advanced protections can be
applied to other classes of bugs, i.e., what is next?  This includes:
input validation, logic problems, leaking secrets, performance degradation,
denial-of-service, command (e.g., SQL) injection, and concurrency attacks.

\subsection{Focus Areas}

\begin{enumerate}
	\item The next generation of vulnerabilities (e.g., logic flaws)
	\item Need good tools for analysis of race conditions / concurrency attacks
\end{enumerate}

\section{Network Layer}

\subsection{Focus Areas}

Discussion centered on new / unexplored ways in which to secure networking
protocols.  One primary area that was discussed centered around
applying artificial diversity to protocols,  with implementation
including potentially taking advantage of the ambiguity of protocols
(i.e., differences in how messages in different protocols are
interpreted by endpoints and middleboxes).  Some group members also
felt like static and dynamic security analysis (to include fuzzing) of
network protocols was an area that needed more work.

\begin{enumerate}
  
  \item Protocol diversity for surviving attacks

  \item Raise the bar on an attacker by forcing them to attack all
  channels to be successful

  \item Protocol dissection is not a solved problem (i.e., how to
  understand and analyze new and unknown protocols)

  \begin{enumerate}
  	\item Profile traffic to application-specific communication for
  	composing into system-specific configuration baseline
  \end{enumerate}

\end{enumerate}

\section{User Layer}

Discussion centered on trust of sensors in a system.  For instance,
what can be done about sensors that lie (e.g., because they have been
compromised), and also how to force a hacker to own all the sensors in
a system in order to truly take control.  Control theory also needs to
integrate cyber security into its processes, including adding threat
model descriptions for control system design, and clean-slate designs
to improve resilience.

\subsection{Focus Areas}

\begin{enumerate}
	\item Control theory research to improve algorithms
	to be resilient against faults caused by cyber attacks
	\item Sensor correlation to survive owning of sensors
	\item Clean slate control theory design for resilience and
	fail-safe recovery (including graceful degradation and restart in the
	event of a cyber attack)
\end{enumerate}
\end{document}
