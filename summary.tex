\documentclass[11pt]{article}

\usepackage{graphicx}
\usepackage{epsfig}
\usepackage{url}
\usepackage[english]{babel}
\usepackage{vmargin}
\usepackage{times}
\usepackage{amssymb}
\usepackage[fleqn]{amsmath}
\usepackage{cite}
\usepackage{titling}
\usepackage{color}


\widowpenalty=10000
\clubpenalty=10000


\begin{document}

\title{Executive Summary: Total Platform Cyber Protection}
\date{}

\maketitle

Traditional security defenses for the computing environment have
focused on security the \emph{border} of the network, as this is the
entry place for most attackers. However, even with sophisticated
border technologies in place, there are will a rash of security
breaches and data compromises. The goal of this work is to survey the
state of security for the entire computing platform, in order to
identify areas that require additional research.

In this work, we analyzed the security research done at the following
layers of the computing platform: hardware, firmware, bus, hypervisor,
operating system, application, network, and user layer.

At the hardware level, the threat is Trojan logic introduced during
design or manufacturing process. While much work has focused on
detecting Trojans using side-channel analysis, we believe that there
exists opportunity for developing the upper layers of the software to
mitigate the problem (assume untrusted hardware), detecting exploits
that utilize the Trojan, and to improve the detection process, both
before and after hardware is fabricated.

At the firmware level, the threat is that firmware has a high
privilege level and is opaque code that can easily hide a Trojan or a
vulnerability. We believe that more research is necessary on
automating the analysis of firmware, including reverse engineering,
transforming firmware, and increasing analysis to support the
concurrent and real-time requirements.

At the bus/interconnect layer, the threat is that any device attached
to the bus can communicate or impersonate any other device on the bus.
While current approaches focus on application-specific monitoring, we
believe that there must be more research on application-agnostic
monitoring, runtime bus filtering/policy enforcement, and device
authentication.

At the hypervisor layer, the threat is that vulnerabilities in the
hypervisor itself can allow a malicious or compromised guest operating
system to exploit the hypervisor and migrate to other guest operating
systems. While much research has been done on this layer, we believe
that there exists significant opportunity to automatically reduce the
attack surface of the hypervisor, increasing assurance of the
hypervisor, and verifying the hypervisor implementation.

At the operating system layer, the issue is that malicious application
can target the operating system to exploit vulnerabilities.


\end{document}
